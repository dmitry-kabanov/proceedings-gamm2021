%%
%% 2015-11-03 -UK- aktualisiert
%% ====================================================================
\documentclass[pamm,a4paper,fleqn]{w-art}
\usepackage{times,cite,w-thm}
\usepackage[T1]{fontenc}
\usepackage[utf8]{inputenc}
%% By default the equations are consecutively numbered. This may be changed by
%% the following command.
%% \numberwithin{equation}{section}
%%
%%
%% The usage of multiple languages is possible.
%% \usepackage{ngerman}% or
%% \usepackage[english,ngerman]{babel}
%% \usepackage[english,french]{babel}
\usepackage{graphicx}

%-------------------------------------------------------------------------------
% Useful mathematical macros
\newcommand{\Data}{\vec{D}}
\newcommand{\DataExt}{\widetilde{\vec{D}}}
\newcommand{\MSE}{\ensuremath{\text{MSE}}}
\newcommand{\T}{\ensuremath{\text{T}}}
\renewcommand{\vec}[1]{\boldsymbol{#1}}
\newcommand{\mat}[1]{\boldsymbol{#1}}
\newcommand{\VTheta}{\ensuremath{\vec{\theta}}}
\newcommand{\VLambda}{\ensuremath{\vec{\lambda}}}
\DeclareMathOperator*{\argmin}{arg\,min}
\newcommand{\R}{\mathbb R}
\newcommand{\UNN}[1][\text{NN}]{u_{#1}}
\newcommand{\FNN}[1][\text{NN}]{f_{#1}}
\newcommand{\NonlinOp}{\mathcal N\!}
\DeclarePairedDelimiter\norm{\lVert}{\rVert}
% -------------------------------------------------------------------------------

%% 
\begin{document}
%% \def\leftmark{Session title}
%%
%%    The information for the title page will be placed between
%%    \begin{document} and \maketitle. The order of most entries
%%    is determined by the class file and cannot be changed by
%%    rearranging them. The maketitle command follows after the
%%    abstract.
%%
%%    The following commands will be updated by the publisher:
%%
%%    \renewcommand{\copyrightyear}{2016}
%%    \DOIsuffix{pamm.20161zzzz}
%%    \Volume{16} 
%%    \Year{2016} 
%%    \pagespan{1}{}
%%
%%    The short title is optional:

\TitleLanguage[EN]
\title[Divergence-free neural networks]{Estimating divergence-free flow via
  neural networks}

%% Please delete not needed author entries.
%% Information for the first author.
\author{\firstname{Dmitry I.} \lastname{Kabanov}\inst{1,}%
\footnote{Corresponding author: e-mail \ElectronicMail{kabanov@uq.rwth-aachen.de}, 
     phone +49\,1517\,087\,66\,15}.} 
%%
%%    Information for the second author
\author{\firstname{Luis} \lastname{Espath}\inst{1}}
%%
%%    Information for the third author
\author{\firstname{Jonas} \lastname{Kiessling}\inst{2}}
\author{\firstname{Raul F.} \lastname{Tempone}\inst{1,3}}

\address[\inst{1}]{\CountryCode[DE]RWTH Aachen University, Pontdriesch 14--16,
  Aachen 52062, Germany}
\address[\inst{2}]{\CountryCode[SE]KTH Royal Institute of Technology,
  SE-100 44 Stockholm, Sweden}
\address[\inst{3}]{\CountryCode[SA]King Abdullah University of Science and
  Technology, Thuwal, 23955-6900, Saudi Arabia}
%%
%%    \dedicatory{This is a dedicatory.}
%%
%%    Abstract is required.
\AbstractLanguage[EN]
\begin{abstract}
We apply neural networks to the problem of estimating wind flow from given
sparse observations.
We assume that the wind speed is low, hence, the flow is incompressible,
namely, divergence-free.
Following modern trend of combining data and models together to obtain
physics-informed neural networks, we reconstruct the flow by training a neural
network in such manner that it not
only matches the observations but also satisfies the incompressibility condition.
This condition is infeasible to satisfy exactly, and to overcome this,
the network training is reduced from constrained optimization
problem to an unconstrained one via
Lagrange relaxation, where the objective function consists of two terms: one is
responsible for matching the observations and another one approximately
satisfies the constraint of flow incompressibility.
The balance between these
two terms is found through a cross-validation procedure.  We apply this
approach to the dataset from windmills in Sweden and compare its effectiveness
with a more traditional technique of flow reconstruction via Fourier
expansions.
\end{abstract}
%% maketitle must follow the abstract.
\maketitle                   % Produces the title.

\section{Introduction}

An important method of finding relations from given dataset of observations
is through fitting a function to this dataset.
The function should be ``rich'' enough so that the relations between independent
and dependent variables in the dataset can be modeled properly.
Recently, special nonlinear functions called neural networks have emerged for
the purpose of finding relations in the data, particularly, an important class
of functions named \emph{physics-informed neural networks} ??? has emerged for
the purposes of fitting the data, for which one can postulate important physical
properties of this data.

We apply physics-informed neural networks to the problem of estimating flows
with divergence-free property, namely, flows of incompressible fluids.
For that, we train the network to not only satisfy the given dataset of
observations but also to approximately satisfy the divergence-free condition.

We demonstrate the performance of such networks on two examples.
One example is based on the truly divergence-free data.
The second example is based on the dataset of wind velocity observations over
Sweden in 2018, where we use an assumption that the wind is incompressible due
to the relatively low wind velocities.

\section{Mathematical model}
Consider dataset
\[
  \vec{D} = \left\{\vec{x}_i, \vec{u}_i\right\}, \quad i = 1, \dots, N
\]
where $\vec{x}_i = (x_i, y_i) \in \mathcal D \subset \R^2$ is a spatial point,
$\vec{u}_i = (u_i, v_i) \in \R^2$ is the velocity field at point $i$.

Assume that the velocity field $\vec u (\vec x)$ satisfies the
divergence-free condition, at least approximately:
\[
  \nabla \cdot \vec u (\vec x) \approx 0.
\]

\textbf{Goal}: estimate the velocity field $\vec u  (\vec x)$ using the above
condition.

Instead of the true velocity field, we seek for estimator
\[
  \hat{\vec u} (\vec x; \vec \theta)
\]
parameterized by
\[
  \vec \theta \in \R^n
\]    
that satisfies
the measurements and the divergence-free condition
\[
  \arg \min_{\vec \theta} \quad 
  \mathbb E \left[ \norm{\vec u(\vec x) - \hat{\vec u}(\vec x; \vec{\theta})}^2_{L^2(\R^2)} \right] +
  \gamma \norm{\nabla \cdot \hat{\vec u}(\vec x; \vec{\theta})}^2_{L^2(\R)} 
\]
where $\gamma$ is a regularization parameter that determines how strongly
the divergence-free condition is enforced.

\[
  \arg \min_\theta \quad 
  \mathbb E \left[ \norm{\vec u(\vec x) - \hat{\vec u}(\vec x; \vec{\theta})}^2_{L^2(\R^2)} \right] +
  \gamma \norm{\nabla \cdot \hat{\vec u}(\vec x; \vec{\theta})}^2_{L^2(\R)} 
\]

Finite number of measurements to evaluate the first term. 

Introduce an additional set of $P$ points, at which
the divergence-free condition is enforced.

Then the problem is to find
\[
  \arg \min_\theta \quad
  \frac{1}{N} \sum_{i=1}^N \norm{\vec u_i - \hat{\vec u}(\vec x_i; \theta)}^2_2
  \ + \;
  \frac{\gamma}{P} \sum_{i=1}^P \left( \nabla \cdot \hat{\vec u}(\vec x_i; \theta)\right)^2
\]



\begin{acknowledgement}
  An acknowledgement may be placed at the end of the article.
\end{acknowledgement}

\vspace{\baselineskip}
%% The style of the following references should be used in all documents.

\begin{thebibliography}{1}

\bibitem{bib1}% 
 F.\,M. Firstauthorfamilyname, F.\,M. Secondauthorfamilyname, and
  C.~Lastauthorfamilyname,
 Abbreviatedjournalname \textbf{volume}, page (year).

\bibitem{bib2}% 
 F.~Examplename and  I.\,E. Anotherauthorname,
 phys. stat. sol. (a) \textbf{1}, 111 (2050).

\bibitem{bib3}% 
 A.~Firstauthorname,  B.~Secondauthorname,  and
  C.~Thirdauthorname,
Here Goes the Title of the Book (Publisher, City, year), p.\,111.

\bibitem{bib4}% 
 A.~Firsteditorname,  B.~Secondeditorname,  and
  C.~Thirdeditorname (eds.),
Here Goes the Title of the Edited Book (Wiley-VCH, Berlin, 2050), p.\,111.

\bibitem{bib5}% 
 D.~Contributorname,
 in: The Title of the Book, edited by The Name of the Editors, Followed by
  the Title of the Series of Books (Publisher, City, year), chap.~1.

\bibitem{bib6}% 
 A.~Lastbutnotleastname,
 Proceedings 1st Dummy Conference on Citation Formatting, City,
  Country, Part A (Publisher, City, year),  pp.\,1--11.

\end{thebibliography}

\end{document}               % End of document.
%%
%% End of file `pamm-stpl.tex'.
